% CATEGORY THEORY 2026
% Abstract template
% See https://ct2026.com/ct/submit/ for instructions
% 1. one-page limit
% 2. avoid loading additional packages

\documentclass[10pt,letterpaper]{article}
\usepackage[margin=1in]{geometry}
\usepackage[T1]{fontenc}
\usepackage{amssymb}
\usepackage{amsthm}
\usepackage{mathtools}
\usepackage{tikz-cd} % for commutative diagrams
\usepackage{etoolbox}\patchcmd{\thebibliography}{\section*{\refname}}{}{}{}

% WARNING: Hyperlinks will be stripped from PDF program booklet and will only appear on web.
\usepackage[allcolors=black,pdfborder={0 0 0}]{hyperref}

\newcommand{\coauthors}[1]{\par\noindent{\small\textbf{Joint work with:} #1}\par\medskip}

\begin{document}
\pagenumbering{gobble}

\title{Functorial Programming for $\omega$-Categories in Lambdapi:\\
Internal Dependent Homs and Computational Adjunctions}
% PRESENTER NAME ONLY - do not list coauthors here
\author{m--- / emdash project}
% LEAVE THE DATE BLANK
\date{}\maketitle

We report on an ongoing experiment, \texttt{emdash2.lp}, aimed at an \emph{internal} and
\emph{computational} specification of (strict/lax) $\omega$-categorical structures inside the
Lambdapi logical framework \cite{lambdapi}. The guiding principle is ``functorial programming'' in the
sense that typing expresses variance and higher coherence, while normalization is driven by a
carefully oriented rewrite/unification layer. This perspective is inspired by the proof-theoretic
reading of categorical structure advocated by Do{\v{s}}en and Petri{\'c} \cite{dosen-petric99}.

A recurring technical obstacle for $\omega$-categorical computation is \emph{iteration}: operations on
cells should not collapse higher structure prematurely. In \texttt{emdash2}, cells are represented
relatively (an $(n{+}1)$-cell is a relative $1$-cell in an iterated hom-category), and rewrite rules
are designed to preserve ``functor-headed'' normal forms whenever further iteration may be needed.
This is reflected in an explicit split between (i) pointwise evaluation symbols and (ii) packaged
``stable heads'' whose $\beta$-rules unfold only when a concrete cell is supplied.

Within this discipline, we emphasize a simplicial/Grothendieck presentation of higher data.
The internal covariant hom construction
\texttt{hom\_cov\_int} packages the bifunctorial hom as a functor
$A^{\mathrm{op}}\to (B\to \mathbf{Cat})$.
Its dependent analogue \texttt{homd\_cov\_int} (built from a base \texttt{homd\_cov\_int\_base})
internalizes a ``triangle/surface classifier'' for displayed categories:
it organizes $2$-cell-like data as morphisms \emph{over a base arrow} (the start of a $3$-simplex),
while remaining amenable to definitional computation in the Grothendieck/Grothendieck case.
This makes it possible to express and iterate ``components of components'' uniformly, via the
\texttt{tapp1*}/\texttt{tdapp1*} projection packagings for transfors and displayed transfors, and the
component extractors \texttt{tapp0\_fapp0}/\texttt{tdapp0\_fapp0}.

As an application, we outline a computational treatment of adjunctions in which unit and counit are
transfors and the triangle identities are oriented as \emph{cut-elimination} steps on composites.
Concretely, \texttt{emdash2} defines an adjunction datum \texttt{adj} together with arrow-indexed
components (via \texttt{tapp1\_fapp0}), and includes a prototype ``triangle cut'' rewrite
implementing the reduction
\[
\varepsilon_f \circ L(\eta_g)\;\;\rightsquigarrow\;\; f \circ L(g),
\]
mirroring the eliminations studied in \cite{dosen-petric99}. The long-term goal is to extend this
normalization story from the $2$-categorical core to the full $\omega$-setting by replacing strict
naturality postulates with higher (simplicial) data mediated by \texttt{homd\_cov\_int}.

% DELETE IF NOT INCLUDING REFERENCES
\begin{thebibliography}{1}

\bibitem{dosen-petric99}
K.~Do{\v{s}}en and Z.~Petri{\'c},
\emph{Cut-Elimination in Categories},
Kluwer Academic Publishers, 1999.

\bibitem{lambdapi}
F.~Blanqui \emph{et al.},
\emph{The Lambdapi Logical Framework},
\href{https://github.com/Deducteam/lambdapi}{https://github.com/Deducteam/lambdapi}.

\end{thebibliography}

\end{document}
