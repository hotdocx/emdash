% CATEGORY THEORY 2026
% Abstract template
% See https://ct2026.com/ct/submit/ for instructions
% 1. one-page limit
% 2. avoid loading additional packages

\documentclass[10pt,letterpaper]{article}
\usepackage[margin=1in]{geometry}
\usepackage[T1]{fontenc}
\usepackage{amssymb}
\usepackage{amsthm}
\usepackage{mathtools}
\usepackage{tikz-cd} % for commutative diagrams
\usepackage{etoolbox}\patchcmd{\thebibliography}{\section*{\refname}}{}{}{}

% WARNING: Hyperlinks will be stripped from PDF program booklet and will only appear on web.
\usepackage[allcolors=black,pdfborder={0 0 0}]{hyperref}

\newcommand{\coauthors}[1]{\par\noindent{\small\textbf{Joint work with:} #1}\par\medskip}

\begin{document}
\pagenumbering{gobble}

\title{Functorial programming for strict/lax $\omega$-categories in Lambdapi}
% PRESENTER NAME ONLY - do not list coauthors here
\author{m--- / emdash project}
% LEAVE THE DATE BLANK
\date{}\maketitle

We report on an ongoing experiment, \texttt{emdash2.lp}, whose goal is a new \emph{type-theoretical}
definition of $\omega$-categories that is both \emph{internal} (expressed inside dependent type
theory) and \emph{computational} (amenable to normalization by rewriting).
The implementation target is the Lambdapi logical framework \cite{lambdapi}, and the broader
methodological inspiration is the proof-theoretic viewpoint that categorical equalities should be
presented as normalization steps (``cut-elimination'') \cite{dosen-petric99}.
In this sense, the Lambdapi specification itself already behaves like a small programming
language/proof assistant for $\omega$-categories: the theory is formulated internally and computations
are performed by normalization; what remains is chiefly elaboration and user-facing syntax (see also
\href{https://github.com/hotdocx/emdash}{https://github.com/hotdocx/emdash}).

The central construction is inspired by the recent ``bridge type'' approach to internal parametricity
without an interval \cite{altenkirch-chamoun-kaposi-shulman24} and, at the same time, by
type-theoretic accounts of weak higher categories \cite{finster-mimram17}.
Instead of starting from a globular presentation and then postulating a large family of coherence
cells, we use a \emph{dependent comma/arrow (dependent hom) construction} that directly organizes
``cells over a base arrow'' in a simplicial manner.

Concretely, let $B$ be a category and let $E$ be a dependent category over $B$ (informally, a fibration
$E\colon B\to \mathbf{Cat}$). Fix a base object $b\in B$ and a fibre object $e\in E(b)$.
We construct a Cat-valued functor that assigns to a base arrow $f:b\to b'$ and a fibre object
$x\in E(b')$ a category of ``morphisms from the transport of $e$ along $f$ to $x$ in the fibre over
$b'$''. In slogan form, this is a dependent arrow/comma object
\[
\mathrm{Homd}_E(e,-)\;:\; E \times \bigl(\mathrm{Hom}_B(b,-)\bigr)^{\mathrm{op}} \longrightarrow \mathbf{Cat}.
\]
This construction plays the role of a bridge/logical-relations principle for fibrations:
it internalizes the idea that higher structure can be accessed by varying along a base morphism and
tracking how fibrewise homs change. Iterating it yields a \emph{simplicial} presentation of higher
cells (triangles, ``surfaces'', and higher simplices) that makes horizontal ``stacking'' of $2$-cells
along a $1$-cell particularly direct.
In particular, ``stacking'' is organized by $2$-cells living over a chosen base edge:
{\scriptsize
\[
\begin{tikzcd}
	&&&&& {1 \ \bullet} \\
	\\
	\\
	\\
	{0 \ \bullet} &&&&& {\bullet \ 2} \\
	\\
	\\
	&&&&&&&& {\bullet \ 3}
	\arrow["b"{description}, from=1-6, to=5-6]
	\arrow[""{name=0, anchor=center, inner sep=0}, "{b'}"{description}, dashed, from=1-6, to=8-9]
	\arrow[""{name=1, anchor=center, inner sep=0}, "f"{description}, from=5-1, to=1-6]
	\arrow[""{name=2, anchor=center, inner sep=0}, "{f'}"{description}, from=5-1, to=5-6]
	\arrow[""{name=3, anchor=center, inner sep=0}, "{f''}"{description}, dashed, from=5-1, to=8-9]
	\arrow[""{name=4, anchor=center, inner sep=0}, "t"{description}, dashed, from=5-6, to=8-9]
	\arrow["{\beta^{op}}", between={0.1}{0.9}, Rightarrow, from=0, to=4]
	\arrow["\sigma"', between={0.1}{0.9}, Rightarrow, from=1, to=2]
	\arrow["\phi"', between={0.2}{0.8}, Rightarrow, from=2, to=3]
\end{tikzcd}
\]
}
Compared with parametricity-based formalisations of semi-simplicial structures, we aim to keep the
interface conceptual and type-theoretic rather than explicitly combinatorial \cite{herbelin-ramachandra24}.

As an application, we outline a computational formulation of adjunctions in which unit and counit are
first-class $2$-cell data and the triangle identities are oriented as cut-elimination reductions on
composites, following \cite{dosen-petric99}. In particular, reductions of the form
\[
\varepsilon_f \circ L(\eta_g)\;\;\rightsquigarrow\;\; f \circ L(g),
\]
are definitional computation steps rather than external equalities. The long-term goal is to develop
the infrastructure/kernel for functorial programming and proving with $\omega$-categories (in many
ways simpler to normalize than the $1$-categorical case), to the point where an AI coding agent such
as ``OpenAI GPT-5.2 Codex'' could automatically formalize a basic category theory textbook.

% DELETE IF NOT INCLUDING REFERENCES
\begin{thebibliography}{1}

\bibitem{dosen-petric99}
K.~Do{\v{s}}en and Z.~Petri{\'c},
\emph{Cut-Elimination in Categories},
Kluwer Academic Publishers, 1999.

\bibitem{altenkirch-chamoun-kaposi-shulman24}
T.~Altenkirch, Y.~Chamoun, A.~Kaposi, and M.~Shulman,
\emph{Internal Parametricity, without an Interval},
Proc.\ ACM Program.\ Lang.\ \textbf{8} (POPL), 2024.
\href{https://arxiv.org/abs/2307.06448}{arXiv:2307.06448}.

\bibitem{finster-mimram17}
E.~Finster and S.~Mimram,
\emph{A Type-Theoretical Definition of Weak $\omega$-Categories},
2017.
\href{https://arxiv.org/abs/1706.02866}{arXiv:1706.02866}.

\bibitem{herbelin-ramachandra24}
H.~Herbelin and R.~Ramachandra,
\emph{A parametricity-based formalization of semi-simplicial and semi-cubical sets},
2024.
\href{https://arxiv.org/abs/2401.00512}{arXiv:2401.00512}.

\bibitem{lambdapi}
F.~Blanqui \emph{et al.},
\emph{The Lambdapi Logical Framework},
\href{https://github.com/Deducteam/lambdapi}{https://github.com/Deducteam/lambdapi}.

\end{thebibliography}

\end{document}
