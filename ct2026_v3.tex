% CATEGORY THEORY 2026
% Abstract template
% See https://ct2026.com/ct/submit/ for instructions
% 1. one-page limit
% 2. avoid loading additional packages

\documentclass[10pt,letterpaper]{article}
\usepackage[margin=1in]{geometry}
\usepackage[T1]{fontenc}
\usepackage{amssymb}
\usepackage{amsthm}
\usepackage{mathtools}
\usepackage{tikz-cd} % for commutative diagrams
\usepackage{etoolbox}\patchcmd{\thebibliography}{\section*{\refname}}{}{}{}

% WARNING: Hyperlinks will be stripped from PDF program booklet and will only appear on web.
\usepackage[allcolors=black,pdfborder={0 0 0}]{hyperref}

\newcommand{\coauthors}[1]{\par\noindent{\small\textbf{Joint work with:} #1}\par\medskip}

\begin{document}
\pagenumbering{gobble}

\title{Functorial programming for strict/lax $\omega$-categories in Lambdapi}
% PRESENTER NAME ONLY - do not list coauthors here
\author{\href{https://github.com/hotdocx/emdash}{https://github.com/hotdocx/emdash}}
% LEAVE THE DATE BLANK
\date{}\maketitle

We report on an ongoing experiment, \texttt{emdash version 2}, whose goal is a new \emph{type-theoretical}
definition of $\omega$-categories that is both \emph{internal} (expressed inside dependent type
theory) and \emph{computational} (amenable to normalization by rewriting).
The implementation target is the Lambdapi logical framework \cite{lambdapi}, following \cite{dosen-petric99}.
In this sense, the Lambdapi specification itself already behaves like a small programming
language/proof assistant for $\omega$-categories: the theory is formulated internally and computations
are performed by normalization; what remains is chiefly elaboration and user-facing syntax in TypeScript.

The central construction is similar to the recent ``bridge type'' approach to internal parametricity
without an interval \cite{altenkirch-chamoun-kaposi-shulman24} and, at the same time, by
type-theoretic accounts of weak higher categories \cite{finster-mimram17}.
We use a \emph{dependent hom / comma / arrow construction} that directly organizes
``cells over a base arrow'' in a simplicial manner.

Concretely, let $B$ be a category and let $E$ be a dependent category over $B$ (informally, a fibration
$E\colon B\to \mathbf{Cat}$, or a more general isofibration). Fix a base object $b_0\in B$ and a fibre object $e_0\in E(b_0)$.
We construct a Cat-valued functor that assigns to a base arrow $b_{01}:b_0\to b_1$ and a fibre object
$e_1\in E(b_1)$ a category of ``morphisms from the transport of $e_0$ along $b_{01}$ to $e_1$ in the fibre over
$b_1$'':
\[
\mathrm{Homd}_E(e_0,(-,-))\;:\; E \times_B \bigl(\mathrm{Hom}_B(b_0,-)\bigr)^{\mathrm{op}} \longrightarrow \mathbf{Cat}.
\]
\[
\mathrm{Homd}_E(e_0,(e_1,b_{01}))\;\coloneqq\;\mathrm{Hom}_{E(b_1)}\bigl((b_{01})!\,e_0,\;e_1\bigr),
\]
In \emph{internalized} syntax notation, this is
\[
\mathrm{Homd}_E\;:\;\Pi b_0,\;E[b_0]^{\mathrm{op}}\to
\bigl(\Sigma b_1,\;E[b_1]\times \mathrm{Hom}_B(b_0,b_1)^{\mathrm{op}}\bigr)\to \mathbf{Cat}.
\]
Iterating it yields a simplicial presentation of higher cells (triangles, ``surfaces'', and higher
simplices) where ``stacking'' of $2$-cells along a $1$-cell is expressed over a chosen base edge:
{\scriptsize
\[\begin{tikzcd}
	&&&&& {b_0 \ \bullet} \\
	\\
	\\
	\\
	\bullet &&&&& {\bullet \ b_1} \\
	\\
	\\
	&&&&&&&& {\bullet \ b_2}
	\arrow["{b_{01}}"{description}, from=1-6, to=5-6]
	\arrow[""{name=0, anchor=center, inner sep=0}, "{b_{02}}"{description}, dashed, from=1-6, to=8-9]
	\arrow[""{name=1, anchor=center, inner sep=0}, "{e_0}"{description}, from=5-1, to=1-6]
	\arrow[""{name=2, anchor=center, inner sep=0}, "{e_1}"{description}, from=5-1, to=5-6]
	\arrow[""{name=3, anchor=center, inner sep=0}, "{e_2}"{description}, dashed, from=5-1, to=8-9]
	\arrow[""{name=4, anchor=center, inner sep=0}, "{b_{12}}"{description}, dashed, from=5-6, to=8-9]
	\arrow["{b_{012}^{op}}", between={0.1}{0.9}, Rightarrow, from=0, to=4]
	\arrow["{e_{01}}"', between={0.1}{0.9}, Rightarrow, from=1, to=2]
	\arrow["{e_{12}}"', between={0.2}{0.8}, Rightarrow, from=2, to=3]
\end{tikzcd}\]
}
We aim to keep the
interface conceptual and type-theoretic rather than explicitly combinatorial \cite{herbelin-ramachandra24}.

As an application, we outline a computational formulation of adjunctions in which unit and counit are
first-class $2$-cell data and the triangle identities are oriented as cut-elimination reductions on
composites, following \cite{dosen-petric99}.
\[
\varepsilon_f \circ L(\eta_g)\;\;\rightsquigarrow\;\; f \circ L(g)
\]

% DELETE IF NOT INCLUDING REFERENCES
\begin{thebibliography}{1}

\bibitem{dosen-petric99}
K.~Do{\v{s}}en and Z.~Petri{\'c},
\emph{Cut-Elimination in Categories}.

\bibitem{altenkirch-chamoun-kaposi-shulman24}
T.~Altenkirch, Y.~Chamoun, A.~Kaposi, and M.~Shulman,
\emph{Internal Parametricity, without an Interval}.

\bibitem{finster-mimram17}
E.~Finster and S.~Mimram,
\emph{A Type-Theoretical Definition of Weak $\omega$-Categories}.

\bibitem{herbelin-ramachandra24}
H.~Herbelin and R.~Ramachandra,
\emph{A parametricity-based formalization of semi-simplicial and semi-cubical sets}.

\bibitem{lambdapi}
F.~Blanqui \emph{et al.},
\emph{The Lambdapi Logical Framework},
\href{https://github.com/Deducteam/lambdapi}{https://github.com/Deducteam/lambdapi}.

\end{thebibliography}

\end{document}
